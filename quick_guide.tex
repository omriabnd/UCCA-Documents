\documentclass[11pt]{article}
\usepackage[top=0.5in, bottom=1.5in, left=0.5in, right=0.5in]{geometry}

\title{UCCA's Foundational Layer:  Annotation 	Guidelines \\
\normalsize{Version 1.3} \\ 
\url{http://homepages.inf.ed.ac.uk/oabend/ucca.html}}
\author{
Omri Abend and Ari Rappoport
}


\begin{document}

\paragraph{UCCA: General Guidelines.}

\begin{itemize}
\item
Work your way by repeatedly locating Scenes in the text, annotating them and all their sub-units and finding Linkers that link them to other Scenes. 
\item
Every unit in UCCA should be analyzed according to one of the following models, or be left unanalyzable. 
\item
A Function (F) unit does not correspond/introduce a relation or participant. Some structural pattern requires it. If no other category applies, it's a Function. Functions may appear in any place in the hierarchy.
\item
A unit is unanalyzable if none of its sub-units has any semantic input by itself, but only as part of a larger expression. For example: proper names (e.g., ``John Smith''), phrasal verbs (e.g., ``give up'') or fixed expressions (e.g., ``as well as'').
\end{itemize}

\paragraph{Analyzing Scenes.}

\begin{enumerate}
\item
A Scene is what we conceive at a given time. A conception of a state, action or movement. It usually has a specified time and location.
\item
Each scene has exactly one main relation. The main relation tells you what the scene is about, what's happening in it. Process (P) is the main relation of a scene that that develops in time (action or movement). State (S) is the main relation of a scene that persists in time.
\item
A scene may have any number of Participants (A), that is, entities that take part in the Scene. Participants include concrete and abstract entities, locations and in some cases entire Scenes. 
\item
A Scene may have any number of Adverbials (D), which are secondary relations in the Scene. They do not introduce another Scene, but semantically modify it. Common cases of modifications are modal relations (like ``can'', ``may'' or ``want''), manner relations (like ``quickly'' or ``patiently'') and relations that specify a sub-event (like ``begin'' or ``finish'').
\item
Time (T) units specify the time in which the Scene or some part of it happened.
\item
A Scene may contain a Ground relation (G) units, whose primary purpose is to relate the Scene with the speaker, the hearer or the general context in which the text was uttered/written/conceived.
\end{enumerate}

\paragraph{Analyzing Non-Scenes.}

First, decide whether there is only one Center (C), or more than one. A Center is a sub-unit which defines the type of the entity, and is necessary for the conceptualization of the unit.

\begin{enumerate}
\item
{\bf One Center:} Covers most cases. The rest of the units can be  (1) Elaborators (E), if they add more information to the Center or modify it in some way, or (2) Relators (R), mostly in the case of prepositions.
\item
{\bf Two or more Centers:} The unit may include Connectors (N), which connect the Centers by highlighting a common feature (usually "and" or "or").
\end{enumerate}

\paragraph{Analyzing Inter-Scene Linkage.}

There are three possible relations between Scenes:
(1) A Scene may serve as Elaborator (E) in another unit (e.g., ``The dog {\bf that ate my homework}''), 
or (2) as a Participant (A) in another unit (e.g., ``{\bf John's kick} saved the game'').
All other cases are considered Parallel Scenes (H). Hs may be related through Linkers (L) (e.g., ``because'', ``after'', ``and'').






\end{document}

